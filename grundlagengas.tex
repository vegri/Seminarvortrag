\subsection{Grundlagen}

\begin{frame}{Grundlagen}
	\begin{block}{Reaktionen von Strahlung mit Gasmolekülen}
		\begin{itemize}
		  \item Anregung:	$X+p\rightarrow X^*+p$\\
		  \item Ionisation:	$X+p\rightarrow X^++p+e^-$\\
		\end{itemize}		
	\end{block}

	\begin{block}{Arten der Ionisation}
		\begin{itemize}
		  \item primäre Ionisation: $e^-$/Ionen-Paare erzeugt durch eintreffende Strahlung
		  \item sekundäre Ionisation: freie $e^-$, die weitere $e^-$/Ionen-Paare erzeugen
		  etc.
		\end{itemize}
	\end{block}	
	
	\begin{block}{Warum Gas zum Teilchennachweis?}
		 hohe Mobilität der durch eintreffende Strahlung erzeugten Ionen und Elektronen
	\end{block}
\end{frame}


\begin{frame}{Grundlagen}
	\begin{block}{Mittlere Energie für Ionisation}
		\begin{itemize}
		  \item Energie, um $e^-$/Ionen-Paar zu erzeugen: ca. 30 eV
		  \item kaum abhängig von Teilchentyp und Gas
		\end{itemize}
	\end{block}
	\vspace*{0.7cm}
	Um $e^-$/Ionen-Paare zu sammeln, müssen beide lange genug frei sein
	\begin{block}{Hindernisse}
			\begin{itemize}
		  \item Rekombination unter Aussendung eines Photons
		  \item Elektronenbindung: Einfangen freier $e^-$ durch elektronegative Atome unter Aussendung
		  eines Photons
		\end{itemize}
	\end{block}
\end{frame}

\begin{frame}{Transport}
	\begin{block}{Diffusion}
	\begin{itemize}
		  \item ohne $E$-Feld: $e^-$/Ionen-Paare diffundieren vom Ort des Entstehens weg $\rightarrow$
		  Verlust an Energie durch Zusammenstöße $\rightarrow$ therm. Gleichgewicht $\rightarrow$
		  Rekombination/Elektronanlagerung
		  \item Geschwindigkeit im therm. Gleichgewicht: Maxwell-Verteilung\\
		  		$v=\sqrt{\frac{8kT}{\pi m}}$~~~~~~~~~~~~~~~~~~~~~bei $T=300$~K: ca. $v_{e^-}=10^6$~cm/s
% 		  \item Diffusionskoeffizient: $D=\frac{1}{3}\cdot v\cdot \lambda$
% 		 		$\rightarrow$ je höher $D$, desto höher die Ausbreitung pro Zeit $t$
		\end{itemize}
	\end{block}
	
	\begin{block}{Drift}
	\begin{itemize}
		  \item mit $E$-Feld: $e^-$/Ionen-Paare beschleunigen entlang der Feldlinien
		  \item Beschleunigung unterbrochen durch Kollisionen mit Gasmolekülen\\
		  		$\rightarrow$ Geschwindigkeitszunahme begrenzt!
		  \item mittlere Geschw.: \textbf{Driftgeschwindigkeit $u$}
		  \item $u=\mu E$~~~~~~~~~~~~~~~~~~~~~~~~~~bei $T=300$~K: ca. $v_{e^-}=10^6$~cm/s
		  %mit~~~$\mu=\frac{D\cdot e}{kT}$
		\end{itemize}
	\end{block}
\end{frame}

