
\subsection{Definition}
%___________________________________________________________________________________________________

\begin{frame}{Teilchendetektoren}
	\begin{description}
	  \item[Teilchendetektoren] Dienen dem Nachweis freier Teilchen durch Messung
	  verschiedener Parameter
	\end{description}
	
	\begin{block}{Häufig gemessene Eigenschaften}
		\begin{itemize}\setlength{\itemsep}{+5pt}
		  \item Geschwindigkeit
		  \item Impuls
		  \item Energie
		  \item Trajektorie
		  \item Zeit (Synchronisierung mit anderen Detektoren)
		  \item Ladung
		\end{itemize}
	\end{block}
\end{frame}

%___________________________________________________________________________________________________

	\begin{frame}{Spurdetektoren}
	\begin{description}
	  \item[Spurdetektoren] Bestimmen die Trajektorie eines Teilchens durch
	  Detektion an verschiedenen Orten und anschließender Rekonstruktion der
	  Bahnkurve
	\end{description}
	\begin{block}{Wozu die Trajektorie bestimmen?}
		\begin{itemize}\setlength{\itemsep}{+5pt}
		  \item Ursprung des Teilchens zurückverfolgen
		  	\begin{itemize}\setlength{\itemsep}{+5pt}
		    	\item Stammt Teilchen vom
		    	Kollisionspunkt$\rightarrow$Untergrundunterdrückung
		  	\end{itemize}
		  \item Messungen in Kombination mit Magnetfeld:
		   	\begin{itemize}\setlength{\itemsep}{+5pt}
		    	\item Ladung
		    	\item Impuls
		  	\end{itemize}
		\end{itemize}
	\end{block}
\end{frame}

%___________________________________________________________________________________________________
%___________________________________________________________________________________________________

\subsection{Motivation}

%___________________________________________________________________________________________________

\begin{frame}{Pile-Up}
	An LHC-Experimenten wie ATLAS werden zukünftig alle 25 ns ca. 50
	Proton-Proton-Kollisionen gleichzeitig gemessen
	% http://iopscience.iop.org/1742-6596/513/2/022024/pdf/1742-6596_513_2_022024.pdf
	
	\begin{figure}[htp]
	\begin{center}
	  \includegraphics[width=\textwidth]{pileup.jpg}
	  \caption{Simulierter Pile-Up am CMS Detektor [cpu]}
	\end{center}
	\end{figure}
	\vspace{-0.7cm}
	
	\begin{block}{Jede Kollision muss isoliert analysiert werden (E- und
	p-Erhaltung)} Sekundärteilchen müssen je einer p-p-Kollision zugewiesen werden 
	\end{block}
	
	\begin{exampleblock}{Spurrekonstruktion nahe der Kollision}
		Schnittpunkt zwischen rekonstruierter Spur und Strahl stellt Ort der Kollision
		dar
	\end{exampleblock}
\end{frame}

%___________________________________________________________________________________________________


