
\subsection{Grundlagen}


\begin{frame}{Lorentzkraft}
	\begin{block}{Kraft auf geladene Teilchen bei magnetischer Flussdichte
	$\vec{B}$}
		$\vec{F}_L = q \cdot (\vec{v} \times \vec{B})$
	\end{block}

	\begin{block}{$\vec{F}_L$ führt zu Kreisbahn mit radius $r$}
		\begin{itemize}\setlength{\itemsep}{+5pt}
		  \item $r = \frac{m \cdot v}{q \cdot B} \propto \frac{p}{q}$
		  \item Vorzeichen der Krümmung ergibt Vorzeichen der Ladung $q$
		\end{itemize}
	\end{block}
	\begin{exampleblock}{Häufig ist Ladung erlaubter Sekundärteilchen eingeschränkt}
		$q \in \{- e, 0, +e\}$
		$\Rightarrow p = \pm e \cdot r \cdot B$
	\end{exampleblock}
\end{frame}

%___________________________________________________________________________________________________

\begin{frame}{Energieverlust in Materie}
	\begin{block}{Teilchendetektion $\Leftrightarrow$ Energieverlust im Detektor}
		Geladene und ungeladene Teilchen übertragen einen Teil ihrer Energie an die
		durchdrungene Materie	
	\end{block}
	
	
	% http://qgp.uni-muenster.de/~jowessel/pages/teaching/ss03/seminar/reiter.pdf
	\begin{block}{Viele verschiedene mögliche Mechanismen}
		\begin{itemize}
		  \item Bethe-Bloch-Formel
		  \item Rutherford-Streuung
		  \item Tscherenkow-Strahlung
		  \item Bremsstrahlung
		  \item Photoelektrischer Effekt
		  \item Compton-Streuung
		  \item Paarbildung
		  \item \ldots
		\end{itemize}
	\end{block}
\end{frame}

%___________________________________________________________________________________________________

\begin{frame}{Spurdetektoren}
	\begin{block}{Ziel von Teilchendetektoren}
		Absorbierte Energie in auswertbares Signal umwandeln 
	\end{block}
	\vspace{1cm}
	\begin{exampleblock}{Ziel der Spurdetektoren}
		Signal muss Ortsinformation beinhalten
	\end{exampleblock}
\end{frame}

