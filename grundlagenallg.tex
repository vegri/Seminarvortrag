
\subsection[]{Definition \& Grundlagen}




%___________________________________________________________________________________________________

\begin{frame}{Energieverlust in Materie}
	\begin{block}{Teilchendetektion $\Leftrightarrow$ Energieverlust im Detektor}
		Geladene und ungeladene Teilchen übertragen einen Teil ihrer Energie an die
		durchdrungene Materie	
	\end{block}
	
	
	% http://qgp.uni-muenster.de/~jowessel/pages/teaching/ss03/seminar/reiter.pdf
	\begin{block}{Viele verschiedene mögliche Mechanismen}
		\begin{itemize}
		  \item Ionisation
		  \item Rutherford-Streuung
		  \item Tscherenkow-Strahlung
		  \item Bremsstrahlung
		  \item Photoelektrischer Effekt
		  \item Compton-Streuung
		  \item Paarbildung
		  \item \ldots
		\end{itemize}
	\end{block}
\end{frame}

%___________________________________________________________________________________________________

\begin{frame}{Spurdetektoren}
	\begin{block}{Ziel von Teilchendetektoren}
		Absorbierte Energie in auswertbares Signal umwandeln 
	\end{block}
% 	\vspace{1cm}
	\begin{block}{Ziel der Spurdetektoren}
		Signal muss Ortsinformation beinhalten
	\end{block}
	
	\begin{exampleblock}{Idealer Spurdetektor}
		\begin{itemize}
		  \item möglichst wenig Masse (Vermeidung Vielfachstreuung/wenig Energieverlust)
		  \item gute Ortsauflösung
		  \item gute Zeitauflösung/geringe Totzeit
		  \item flexibel
		  \item günstig 
		\end{itemize}
	\end{exampleblock}
\end{frame}

