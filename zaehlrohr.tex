\subsection[]{Zählrohr}

\begin{frame}{Zählrohr}
    \begin{columns}[T]
    
	    \column{.6\textwidth}
			\begin{figure}[htbp]
			  \centering
			  \includesvg[svgpath=bilder/, width=\textwidth]{zaehlrohr}
			  \caption{Aufbau eines Zählrors [wzr]}
			\end{figure}
			
	    \column{.45\textwidth}
	    	\begin{itemize}
	    	  \item Gasgefülltes Rohr
			  \item Hochspannung zwischen Kathode und Anode
			  \item Ionisierende Strahlung bildet freie Elektronen
			  \item Diese driften zur Anode	
			  \item Stromimpuls wird gemessen
			\end{itemize}
    \end{columns}
\end{frame}

%___________________________________________________________________________________________________

\begin{frame}{Zählrohr-Kennlinie}
	Zählrohrspannung beschleunigt freie Elektronen Richtung Anode \\
	Funktionsweise stark abhängig von Spannung
	
	\begin{figure}[htbp]
	  \centering
	  \includesvg[svgpath=bilder/, height=0.5\textheight]{zaehlrohr-kennlinie}
	  \caption{Kennlinie eines Zählrors [wzk]}
	\end{figure}
	
\end{frame}	

%___________________________________________________________________________________________________

\begin{frame}{Zählrohr-Modi}
	\begin{block}{Geringe Spannung ($\ll 100V$): Rekombination}
		\begin{itemize}
		  	\item $e^-$ können mit Ionen rekombinieren
		  	\item Nur wenige Elektronen erreichen Anode
		  	\item Strompuls abhängig von Abstand der Ionisation
		\end{itemize}
	\end{block}
	
	\begin{block}{Höhere Spannung ($O(100V)$): Ionisationskammer}
		\begin{itemize}
		  	\item Keine Rekombination: Alle $e^-$ erreichen Anode
		  	\item Strompuls proportional zur Anzahl der Elektronen
			\item \textbf{Messung der absorbierten Energie möglich}
		\end{itemize}
	\end{block}
	
\end{frame}

\begin{frame}{Zählrohr-Modi}
	\begin{block}{Höhere Spannung: Proportionalzählrohr}
		\begin{itemize}
		  	\item $e^-$ ionisieren nahe der Anode weitere Atome (Lawinen)
			\item $\Rightarrow$ Verstärkung des Strompulses
		  	\item Signal weiterhin proportional zur absorbierten Energie
		\end{itemize}
	\end{block}
	
	\begin{block}{Höchste Spannung: Geiger-Müller-Bereich}
		\begin{itemize}
	  		\item Energie so hoch, dass UV-Strahlung entsteht, die weitere Stellen
	  		Ionisiert
	  		\item Komplette Ionisierung der Kammer $\Rightarrow$ Gasentladung
	  		\item Keine Energiemessung $\Leftrightarrow$ Reiner Zähler
	  		\item Hohe Totzeit (ca. $100\mu s$)
		\end{itemize}
	\end{block}
	Noch höhere Spannung führt zur spontanen Gasentladung
\end{frame}
