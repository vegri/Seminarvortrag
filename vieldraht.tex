\subsection{Vieldrahtproportionalkammer}


\begin{frame}{Vieldrahtproportionalkammer oder MWPC}
    \begin{columns}[T]
	    \column{.5\textwidth}
			\begin{figure}[htbp]
			  \centering
			  \includesvg[svgpath=bilder/, width=\textwidth]{vieldraht}
			  \caption{Aufbau einer Vieldrahtkammer [wvk]}
			\end{figure}
			
	    \column{.45\textwidth}
	    	\begin{itemize}
	    	  \item Proportionalkammer mit vielen Drähten statt einem
			  \item äquidistante Drähte (Anode) zwischen zwei Platten (Kathode)
			\end{itemize}
			
			\begin{figure}[htbp]
			  \centering
			  \includesvg[svgpath=bilder/, width=\columnwidth*2/3]{vieldrahtfeld}
			  \caption{Feldlinien einer Vieldrahtkammer [wvkf]}
			\end{figure}
    \end{columns}
\end{frame}



\begin{frame}{Vieldrahtproportionalkammer}
    	\begin{block}{Wie kommt die Spur zustande?}
		\begin{itemize}
		  \item $e^-$/Ionen-Paare driften entlang der Feldlinien zur nächsten Anode/Kathode
		  \item $e^-$ lösen bei hoher Feldliniendichte Ladungslawine an Anode aus $\rightarrow$
		  Signal$\rightarrow$ eindimensional e Ortsauflösung
		  \item 2D: zweite Lage von Drähten senkrecht auf erster ($X-Y$ MWPC)
		  \item 3D: mehrere MWPC hintereinander
		\end{itemize}
	\end{block}
	 
\end{frame}
