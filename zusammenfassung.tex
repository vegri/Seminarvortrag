

\begin{frame}{Zusammenfassung und Vergleich}
	
	\begin{block}{Gasdetektoren}
			\begin{itemize}
		  \item Signal: freie $e^-$/Ionen-Paare
		  \item großflächig aufbaubar, günstig, direktes Signal; hohe Ansprechzeit, mäßige Ortsauflösung
		\end{itemize}
	\end{block}
% 		    	\vspace{0.5cm}
	\begin{block}{Halbleiterdetektoren}
			\begin{itemize}
		  \item Signal: freie $e^-$/Loch-Paare
		  \item sehr gute Ortsauflösung, direktes Signal; teuer, hohe Verlustleistung
		\end{itemize}
	\end{block}
% 		    	\vspace{0.5cm}
	\begin{block}{Szintillationszähler}
			\begin{itemize}
		  \item Signal: Photonen
		  \item wenig Masse, kleine Ansprechzeit, flexibel; schlechte Lichtausbeute, Umwandlung Signal
		\end{itemize}
	\end{block}

\end{frame}


\begin{frame}{Was gibt es noch?}
	
	\begin{block}{Detektoren}
		\begin{itemize}
		  \item Liquid Ionisation Detectors (z.B. Blasenkammer)
		  \item TPC (Time Projection Chamber)
		  \item GEMs (Gas Electron Multiplier)
		  \item Szintillatoren (Hodoskope)
		  \item \ldots
		\end{itemize}
	\end{block}
	
		\begin{block}{Spurrekonstruktion}
		\begin{itemize}
		  \item Hough-Transformation
		  \item Kalman-Filter
		\end{itemize}
	\end{block}

\end{frame}