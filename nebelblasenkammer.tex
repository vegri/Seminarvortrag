\subsection{Nebel- und Blasenkammer}

%___________________________________________________________________________________________________

\begin{frame}{Nebelkammer}
	Erfindung 1911 durch Charles Wilson
	\begin{block}{Kammer mit übersättigtem Wasserdampf}
		\begin{itemize}
		  \item Ionisierende Strahlung bildet Ionen-Elektronen-Paare im Gas
		  \item Ionen wirken als Kondensationskeim
		  \item Kondensierte Tröpfchen bilden sichtbaren Kondensstreifen 
		  \item Fotografie der Kondensstreifen für spätere Spurrekonstruktion
		\end{itemize}
	\end{block}
	Später: Übersättigte Luft-Alkohol-Gemische\\
	Heute: Nur noch für demonstrative Zwecke
\end{frame}

%___________________________________________________________________________________________________

\begin{frame}{Blasenkammer}
	Erfindung 1952 durch Donals A. Glaser
	\begin{block}{Ähnliches Prinzip wie Nebelkammer}
		\begin{itemize}
		  \item Metastabiler Zustand von flüssigem Wasserstoff bei adiabatischer
		  Druckänderung
		  \item Ionen wirken als Siedekeim
		  \item Entstandene Bläschen visualisieren Teilchenspur 
		  \item Fotografie der Blasenspuren für spätere Rekonstruktion
		\end{itemize}
	\end{block}
\end{frame}

%___________________________________________________________________________________________________

\begin{frame}{Blasenkammer - Aufbau}
    \begin{columns}[T]
    
	    \column{.6\textwidth}
			\begin{figure}[htbp]
			  \centering
			  \includesvg[width=\textwidth]{Blasenkammer}
			  \caption{Blasenkammer mit B-Feld [wbk]}
			\end{figure}
			
	    \column{.45\textwidth}
	    	\begin{enumerate}
			  \item Kurz vor Messung wird Druck verringert
			  \item Temperatur nun oberhalb des Siedepunktes
			  \item Kameras bilden Blasenspuren ab
			  \item 3D-Rekonstruktion der Spuren möglich
			  \item Druck wird wieder erhöht damit sich Blasen wieder lösen  
			\end{enumerate}
    \end{columns}
\end{frame}

%___________________________________________________________________________________________________

\begin{frame}{Nebel- und Blasenkammern}
    \begin{columns}[T]
		\column{.45\textwidth}
			Vorteile		
			\begin{itemize}
			  \item Bahnen direkt sichtbar (einfache Rekonstruktion)
			  \item Günstig
			  \item Gute Ortsauflösung: $O(1 \mu m)$
			\end{itemize}	
	    \column{.5\textwidth}
	    	Nachteile
	    	\begin{itemize}
			  \item Langsame Bildung der Spur
			  \item Nur Nebelkammer kann kontinuierlich betrieben werden
			  \item Viel Materie (WW mit Teilchen)
			\end{itemize}
    \end{columns}
    \vspace{1cm}
    Nebelkammern nur noch für demonstrative Zwecke, Blasenkammern finden noch
    Einsatz bei Messungen mit geringen Raten (WIMP)
\end{frame}